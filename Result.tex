\chapter{Result}
Här är väl tanken att jag ska skriva lite om hur jag implementerat
och bearbetat allt relevant som har med implementationen att göra. 
Vad jag gjort som blivit bättre än andra lösningar. Vad jag fokuserat
 på (throughput kontra mängd använd hårdvara).

\section{Problems}
Encountered problems:

\begin{itemize}
\item Not possible to get the license for CSA3
\item Small interrest for CSA3
\item Next to no documentation of the CISSA algorithm
\end{itemize}

% Fortsätt fylla i när du stöter på problem

\section{Hardware}
Såhär ser hårdvaran ut i en sjukt snygg bild jag ritat: \newline

\begin{array}{r l c r l}
  Inputs & &  \_\_\_\_\_\_\_ & & Outputs \\
  1 & --| & & | -- & 1 \\
  2 & --| & & | -- & 2 \\
  3 & --| & HW & | -- & 3 \\
  4 & --| & & | -- & 4 \\
  5 & --| & & | -- & 5 \\
  &     &   \_\_\_\_\_\_\_ & & \\
\end{array}

\section{Flow}
Berätta om flödet på implementationen.

\section{Special solutions}
Berätta om hur det fungerar.

\section{How much of the CSA3 standard has been implemented?}
Berätta om vilka delar du realiserat.
