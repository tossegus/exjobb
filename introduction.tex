\chapter{Introduction}
Den här rapporten baseras på ett examensarbete utfört på WISI NORDEN 
AB. Examensarbetet handlade om utvärdering och implementering av 
potentiella scrambling metoder som ersättare åt DVB-CSA och tog plats 
under vårterminen 2014.

Vad ska stå här??

\section{Background}
The formerly used \emph{common scrambling algorithm} (CSA) has due to 
recent progresses in television broadcasting become obsolete. CSA was 
designed to make software descrambling hard, if possible.



There are two suggested replacements of CSA. Those are the software-
friendly CISSA and the hardware-friendly CSA3. Both of them are based 
on the AES-128 algorithm.

Varför gjordes exjobbet?

%UTVECKLA DET HÄR. JAG MINNS INTE EXAKT VAD ANLEDNINGEN VAR. 
%LÄS IGENOM LANTTOS MAIL.

\section{Problem specification}
The task was to analyze what possible replacements existed for the 
common scrambling algorithm, and which one would be the most suitable 
replacement. 
After choosing an algorithm, I was to implement the chosen algorithm.

There were two proposed replacements to be compared and analyzed 
to find what made one of them software-friendly and the other one 
hardware-friendly.

Vad var uppgiften. Frågor att besvara.

%ÄR DET HÄR KOPPLAT TILL NAMNET JAG HAR SKRIVIT PÅ MIN RAPPORT?

\section{Frågeställning}
Kan sammanbindas med problemformuleringen.

JAG FÖRSTÅR INTE VAD DET HÄR ÄR.

\section{Constraints}
The thesis has been limited to implementing the chosen scrambling 
algorithm. The implementation was be optimized towards hardware usage, 
while achieving the desired frequency used in the rest of the 
Field-Programmable Gate Array (FPGA).


This limitation is very diffuse, since there are many 
ways of increasing the speed, while the different ways require more or 
less hardware.


While the entire circuit can be made as a combinatorial circuit, it 
will most likely fill up the entire FPGA.

Hur har exjobbet begränsats? Det har väl knappt funnits begränsningar?

HÄR FÅR DU UTVECKLA LITE. ÄR DET BRA ELLER DÅLIGT ATT HA MED DET HÄR?? 
VAD ÄR DE FAKTISKA BEGRÄNSNINGARNA?

\section{Methodology}
The project was split into a set of tasks, to be performed in the order 
written below. Performing the tasks in this order was done to decrease 
the difficulty of the work.

\begin{itemize}
\item Litterature study
\item Choosing an algorithm
\item Design and test of entities
\item Implementation
\item Optimization
\end{itemize}

I began the research by studying litterature, to find out what 
cryptography was about. This provided some insight into what the 
strenghts and weaknesses the algorithms actually were. This gave me 
some depth, and I chose to start of with a cipher that was actually 
used in both of the algorithms that I studied. Using the gathered 
background information about how the algorithm worked made design 
and testing of the entities rather easy. I initially designed the lower
level entities, which allowed for easier testing of seperate parts of 
the system. Since I already knew that they functioned properly, due to 
low level testing, it was easier to merge them with other entities, to 
build the system bottom-up. 

Hur har exjobbet genomförts? Det här behöver skrivas om och förbättras. 
Det är ju hur jag jobbat, men jag vet inte hur vettigt det jag skrev 
var.

%JAG HAR INGEN ANING.
