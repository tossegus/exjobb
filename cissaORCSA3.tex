\chapter{CISSA or CSA3}\label{ch:csa3cissa}
There are currently two scrambling algorithms being assessed as 
replacements to the currently used DVB-CSA. A replacement is 
needed to assure content security for yet another ten years.
A part of this thesis has been to compare the two proposed 
algorithms and decide which one is the most suitable replacement.
This chapter is a basic introduction to the two algorithms, and 
the algorithm implemented in this thesis is presented.

\section{Replacements}
CISSA is meant to be a hardware-friendly as well as software-friendly 
algorithm designed to allow descrambling to be made on CPU-based units 
such as computers, smart phones and tablets \citep[p. 9]{DVB:2013}.

CSA3 is a hardware-friendly, software-unfriendly scrambling algorithm 
chosen by the ETSI to replace the currently used CSA 
\citep[pp. 6--7]{DVB:2013}. Software-unfriendly means that descrambling 
is designed in such a way that it is highly impractical to perform in 
software.

Both of the algorithms are to be implemented in hardware for 
scrambling of data. The difference is that CSA3 is to make it hard to 
descramble the material using software. Since both of the algorithms 
are confidential, it is sadly impossible to find out what makes the 
CSA3 algorithm software-unfriendly, while the CISSA algorithm is 
software-friendly. 
%%%%%%%%%%%%%%%
%\Warning[Source]{Om jag får be snällt}
%%%%%%%%%%%%%%%

\section{CISSA}
CISSA stands for \emph{Common IPTV Software-oriented Scrambling 
Algorithm} and is designed to be software-friendly. Opposite to the 
CSA3, CISSA is made to be easily descrambled in software, so that 
CPU-based systems such as computers and smart-phones can implement 
it.  Although it is software-friendly, it is supposed to able to 
be implemented efficiently on hardware as well as in software 
\citep[p. 9]{DVB:2013}.

CISSA is to use the AES-128 block cipher in CBC-mode with a 16 byte 
IV with the value 0x445642544d4350544145534349535341. Each TS packet 
is to be processed independently of other TS packets, but each block 
of data in the payload depends on the previous blocks of data in the 
same payload, except the first block of data, which depends on the IV. 
Both the header and adaptation field are to be left unscrambled. 
\citep[p. 11]{DVB:2013}

\subsection{Software friendly}
An FPGA implementation of the CISSA algorithm is implementable, due 
to the fact that the scrambling of the content is supposed to be 
made in hardware, even though the descrambling is supposed to be 
made either in hardware or software.

While having a scrambling algorithm designed to enable viewing on 
CPU-based units opens up the market for more users, it might increase 
the risk for algorithm theft. Since reverse-engineering is possible 
for software implementations, one might find the algorithm for 
descrambling, as well as scrambling through inversion of the algorithm.
Knowing the algorithm enables cryptoanalysists to search for 
weaknesses in the algorithm, with the purpose of breaking it.

"A cryptosystem should be secure even if everything about the system, 
except the key, is public knowledge." according to Kerckhoffs's 
Principle.
%%%%%%%%%%%%
%\Warning[Source]{No no, not Wikipedia}
%%%%%%%%%%%%
This means that the only result of having a descrambling method suited 
for hardware as well as software implementation should possibly only 
result in some free implementations showing up. But it being 
implemented in software should not lead to any problem.

%Is this even a problem? I think WISI utilizes a free implementation of the CSA.

\section{CSA3}
The CSA3 scrambling algorithm is based on a combination of an Advanced 
Encryption Standard (AES) block cipher using a 128-bit key, which is 
called the AES-128, and a confidential block cipher called the XRC 
\citep[p. 8]{DVB:2013}. XRC stands for eXtended emulation Resistant 
Cipher and is a confidential cipher used in DVB \citep[p. 8]{DVB:2013}.

\subsection{Hardware friendly}
The CSA3 is designed to be hardware-friendly, meaning that 
descrambling through software methods is supposed to be next to 
impossible. Using a software-hostile descrambling algorithm means that 
reverse-engineering and algorithm theft becomes hard, if even possible.
Even though it would decrease the probability of content theft, it 
closes the door to expansion onto the CPU-based units market, which is 
becoming larger and larger.

\section{Selection of the algorithm}
CSA3 implements the AES-128 cipher for scrambling, combined with a 
confidential cipher, called the XRC cipher. CISSA does not on the 
other hand contain any confidential cipher. CISSA uses the AES-128 
cipher in CBC-mode with a static IV \citep{DVB:2013}.

CISSA sounds like a great idea, since it would allow CPU-based units 
to descramble data streams without using a dedicated HW-Chip. Regardless
of which cipher is the best, or will prove to become the next standard, 
both of them use AES-128 as a building block. Therefore, starting out 
with an AES-128 chiper provided for a basis to continue developing 
the scrambler towards either CISSA or CSA3 on a later stage. The 
algorithm that was finally chosen was the CISSA algorithm due to three
reasons. Firstly, software descrambling seems to be the future of 
content protection. Secondly, CISSA was a free and open algortihm. 
Finally, AES-128 in CBC mode (which is basically CISSA) is needed in 
order to use CI+ \citep[p. 15]{CI+:2011}.


