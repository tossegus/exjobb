\chapter{CISSA and CSA3}
There are currently two scrambling algorithms being assessed as replacements to 
the currently used DVB-CSA. This is done to assure content security for 
yet another ten years.

%%%%%%%%%%%%%%
%What are the pros and cons of having hardware scramblers and software scramblers.Software uses itself of the date to generate a random key. Hardware uses itself %of additive scramblers, and other random output generators to generate the key. %%%%%%%%%%%%%%

CISSA is meant to be a hardware-friendly as well as software-friendly algorithm 
designed to allow descrambling on CPU-based units such as computers, smart phones
and tablets \citep[p. 9]{DVB:2013}. CISSA 

CSA3 is a hardware-friendly, software-unfriendly scrambling algorithm chosen by 
the ETSI to replace the currently used CSA \citep[pp. 6--7]{DVB:2013}. It uses 
itself of AES blocks (with keys of lengths of either 128, 192 or 256 bits) as 
well as a confidential block cipher called the XRC for scrambling. 
Software-unfriendly means that descrambling is designed so that it is highly 
impractical to perform in software, but is easily done in hardware.

Both of the algorithms are to be implemented in hardware for scrambling of data.
The difference is that CSA3 is to make it hard to descramble the material in 
software. Since both of the algorithms are confidential, it is sadly impossible 
to find out what makes the CSA3 algorithm software-unfriendly, while the CISSA 
algorithm is software-friendly. \Warning[Source]{Om jag får be snällt}

\section{CISSA}
CISSA stands for \emph{Common IPTV Software-oriented Scrambling Algorithm} and 
is designed to be software-friendly. Opposite to the CSA3, CISSA is made to be 
easily descrambled in software, so that CPU-based systems such as computers and 
smart-phones can also implement it.  Although it is software-friendly, it is 
supposed to able to be implemented efficiently on hardware as well as in 
software \citep[p. 9]{DVB:2013}.

CISSA is to use the AES-128 block cipher in CBC-mode with a 16 byte IV with the 
value 0x445642544d4350544145534349535341. \citep[p. 11]{DVB:2013}

Är inte meningen att ens IV ska vara hemlig? Den utgör ju en stor del an shiffret? Om man vet IV borde man kunna inputta en ciphertext som plaintext så att man kan ta reda på vilken det ursprungliga plaintexten var?
\Warning[Find more]{But I don't know the name of the second block cipher}

\subsection{Software friendly}
What makes this algorithm software friendly?
Is it still possible to make use of it on an FPGA - since FPGAs are so
general?

\section{CSA3}
CSA is to be succeeded by CSA3 which is based on a combination of a 128-bit 
AES block cipher, which is simply called the AES-128, and a confidential block 
cipher called the XRC \citep[p. 8]{DVB:2013}.

\subsection{XRC}
XRC stands for eXtended emulation Resistant Cipher and is a confidential cipher 
used in DVB \citep[p. 8]{DVB:2013}.

\subsection{Hardware friendly}
What makes the CSA3 so hardware friendly?
Is it because it is meant to be a secret standard, only to be delivered
on a chip, to make it as good as impossible to reverse-engineer?

\section{Conclusion}
From what I've seenff, both the CISSA and CSA3 implement the AES-128 for 
scrambling, combined with a secret cipher. The secret cipher for CSA3 is the XRC 
cipher, and the secret CISSA cipher is yet to be known FÖR JAG HAR INTE HITTAT DET, MEN DET BETYDER INTE ATT ANDRA INTE VET VAD DET HETER. CISSA sounds like a 
great idea in my opinion, allowing CPU-based units to descramble data streams 
without a dedicated HW-Chip. While that is good and all, CSA3 is a finished 
standard, and will probably be more easily implemented on an FPGA, while CISSA 
seems to be still developed. Starting out with an AES-128 chiper would provide 
for a basis to continue development of the scrambling, either towards the CISSA 
or the CSA3 solution, on a later stage.
