\chapter{Result}
The focus of this thesis has been to minimize the amount of hardware 
usage, while trying to meet the timing constraints provided 
from the rest of the circuit. Reaching a throughput of 1 Gbit/s was 
sufficient for the current design.

The implemented circuit was a scrambler, which can be found in the 
head-end of DVB systems. An analysis of two algorithms was done, and 
the AES128 algorithm, in CBC-mode, was chosen. It corresponds to the 
CISSA algorithm, which is designed to be software-friendly. 

AES128 processes 16 bytes of data in 11 clock pulses with a clock 
frequency of 94MHz, which would correspond roughly to a throughput of 
1.16 Gbits/s. The frequency of the design is further discussed in 
section \ref{sec:c_path}. The scrambler needs to process the key 
first, before being able to scramble data. A keyexpansion takes 
roughly 45 clock pulses, and is only performed when a new key is sent, 
which is very seldom. The scrambler then deals with 16 bytes of data 
on 13 clock pulses, but outputs 1 byte of data per clock cycle. This 
is done so that one byte of data from the scrambled package is read 
into a register on every clock pulse. When four bytes are collected 
the 32-bit output is sent out. 32-bits are processed at a time, since 
the data-bus is a 32-bit bus.

\section{Problems}
The main problems encoutered were:

\begin{itemize}
\item Not possible to get the license for CSA3.
\item Small interrest in CSA3 from customers.
\item Next to no documentation of the CISSA algorithm.
\item Hard finding reliable test vectors.
\item Merging.
\item Timing.
\end{itemize}

\subsubsection{License and interrest of the CSA3}
When the Thesis was first started, the idea was that the CSA3 
algorithm was to be implemented. However, licensing problems, and the 
fact that AES-128 in CBC-mode seemed like a better idea, led to a 
rework of the project. Also, the interrest in CSA3 from potential 
customers was small, while the interrest in AES128 in CBC-mode was 
great.

\subsubsection{Documentation of CISSA}
The only mention of the CISSA algorithm found in official documents 
has been found in \citet{DVB:2013}. What can be found from this 
documentation is that CISSA uses the AES128 cipher in CBC-mode 
encyption, with a static IV. While this has been sufficient to 
implement the algorithm, more documentation would have been good.

\subsubsection{Test vectors}
Finding test vectors for the different blocks of the scrambler has 
been hard. It has been possible to find a number of them through the 
ETSI documentation though, which has been used to determine the 
functionality of the blocks.

\subsubsection{Merging and timing} \label{ssec:merge}
Merging has required more focus than expected, probably since this 
design was a bottoms-up design, instead of the more common top-down 
design. This project was done by implementing low level entities 
first, that were to be used in higher hierarchies. Doing this caused 
some problems when merging entities into higher level blocks, since 
some signals, needed to be produced. This was not a huge problem, and 
only occurred on a few instances, but were rather troublesome at those 
times.

The pro of this method has been that it produced results quickly. The 
con is that a large portion of the time has been spent on going back 
to entities that were already functional, and reworking them by adding 
signals, and finding the right timing conditions to make sure that 
they provided nescessary information for entities higher up in the 
hierarchy.

Since the plan was to optimize this implementation to just meet the 
demands on speed, while trying to minimize the amount of hardware 
needed, timing was introduced into a few circuits that could have 
otherwise been completely combinatorial. This has, as expected, 
introduced quite a bunch of timing-issues. All of them appear to be 
gone now. It is however hard to know, without performing more 
exhaustive testing of the system.

The solution described in this thesis includes the entire hardware 
usage, which includes the interface towards the FPGA, which is one of 
the reasons why it might appear large, when compared to other 
implementations.

\section{Hardware}
The top entity can be viewed in Figure \ref{b:scr}, and figures of the 
rest of the entities are placed near the explanation of the entities 
in the following sections.

\begin{figure}
  \includegraphics[width=\textwidth]{scrambler}
  \caption{The top entity}
  \label{b:scr}
\end{figure}

\subsection{Hardware usage} \label{ssec:hu}
The circuit that was first implemented has during the course of the 
project been optimized in a couple of ways it was synthesized. 
It has either been optimized to minimize hardware usage, or maximize 
the speed of the circuit. A total of six synthesises were run. A 
table displaying the differences of the results can be viewed in 
figure \ref{table:synthesis}. Most focus has been put on minimizing the 
amount of registers. This has been done by adding control signals, that 
decrease the need to store values in registers. The most significant 
optimizations are discussed in this section. 

\subsubsection{Optimizations}
The most significant optimization that was performed was an 
optimization of keyblock3. Keyblock3 is located in the keyexpansion 
entity which is explained in section \ref{sec:Expansion} and can be 
seen in Figure \ref{block:keyexpansion}. That entity used a lot of 
the provided resources. It was noticed that the circuit waited for 
the expanded key to become completely filled before updating the 
output at one point. The expanded key was stored in a vector located 
in keyblock3 while it was being updated. However, the expanded key 
used by other entities was not updated until the entire expanded key 
located in keyblock 3 was completely filled. Rewriting the code 
decreased the amount of registers by 176 8-bit registers, which could 
be removed by adding an enable signal. The enable signal tells the 
other entities when the expanded key is ready for usage.

%% \subsubsection{First synthesis}
%% This was the initial synthesis performed on the FPGA. This did not 
%% hold any interresting information, except indications about some 
%% minor errors that could be found in the initial version of the code.
%% The circuit used up 15\% of the FPGA, and had quite a large amount of
%% unnescessary latches and FFs included. It used 11830 FFs, and roughly
%% 3000 latches. The circuit was redesigned to remove the latches and 
%% ran the second synthesis.
%% \subsubsection{Second synthesis}
%% The circuit used up 8\% of the FPGA this time. 
%% What was noticed from the second synthesis was that the keyblock3 
%% entity, as well as keyblock1 entity seemed to use a lot of registers 
%% and multiplexers. It should be possible to replace them with RAMs or 
%% LUTs.
%% The maximum frequency obtained was 92MHz after this synthesis, 
%% where a lot of the minimum period was due to the routing of the 
%% circuit, which  made up for 75\% of the period.
%% To be able to compare the third synthesis with this one, you need to 
%% know that the keyexpansion entity used 1538 D-type FFs and 16 
%% Comparators. It used up 176 8-bit registers, which were used by the 
%% expanded key.
%% The entire circuit still used up roughly 8\% of the FPGA. The 
%% keyexpansion entity used 130 D-type FFs and 16 Comparators this 
%% time.

%% \subsubsection{Third synthesis}
%% The maximum frequency obtained this time was 94MHz. The number of 
%% Slice Registers went down from 4357 to 2945, and the percentage of 
%% Slice Registers decreased from 3\% usage to 2\% usage. The keyblock3 
%% module seems to be using the most hardware from what can be seen. 
%% It uses roughly 1302 multiplexers, which should be reducable.

%% \subsubsection{Fourth synthesis}
%% The next synthesis was made after the state2data module was 
%% rewritten, to remove yet another signal. This should have decreased 
%% the design by a 128-bit register. It was mostly done to try to allow 
%% for a synthesis of the module, while also reducing hardware usage. 
%% This would decrease the number of D-type FFs by yet another 128. A 
%% comparison between report 3 and 4 displays a decrease from 130 D-type
%% FFs to 2 D-type FFs,  which confirms that 128 FFs were removed.
%% The maximum frequency obtained this time was 94MHz. The number of 
%% FFs went from 2945 to 2817. However, this did not affect the number 
%% of Slice Registers.
%% The circuit still uses roughly 8\% of the hardware on the FPGA.

%% \subsubsection{Fifth synthesis}
%% During the fifth synthesis, it was noted that two reports were made 
%% during synthesis and optimization. Those were a Synthesis Report as 
%% well as a Place and Route Report. The ones that have been examined 
%% this far have been the Synthesis Reports, and to make sure that 
%% there are not any huge gaps in numbers between the reports, they wil 
%% be the ones analyzed, and not the Place and Route Reports. Many of 
%% the entities can not be mapped seperately, due to the amount of IOs 
%% on the  FPGA, compared to the number of IOs required by the modules.

The third synthesis was performed on each block seperately, to find 
out where further optimization would yield the biggest improvement. 
The hardware usage can be viewed in Table \ref{synt:fifth}.

%
\newcommand{\MyIndent}{\hspace*{0.2cm}}%

\begin{longtable}{ l | c | c }
  Entity & Slice LUTs out of 63288 & Slice Registers out of 
  126576 \\ \hline
  scrambler              & 5167 & 2817 \\
  \MyIndent \triangleright manager      &  858 &  699 \\ 
  \MyIndent \triangleright cbc          & 4321 & 2127 \\

  \MyIndent \MyIndent \triangleright cipher       & 4229 & 1994 \\

  \MyIndent \MyIndent \MyIndent \triangleright keyexpansion & 2914 
  & 1601 \\

  \MyIndent \MyIndent \MyIndent \MyIndent \triangleright keyblock1 
  & 689 & 0 \\
  \MyIndent \MyIndent \MyIndent \MyIndent \triangleright keyblock2 
  & 208 & 9 \\

  \MyIndent \MyIndent \MyIndent \MyIndent \MyIndent \triangleright 
  demux & 32   & 0 \\
  \MyIndent \MyIndent \MyIndent \MyIndent \MyIndent \triangleright 
  keycore & 183 & 9 \\

  \MyIndent \MyIndent \MyIndent \MyIndent \MyIndent \MyIndent 
  \triangleright ctr & 14 & 9 \\
  \MyIndent \MyIndent \MyIndent \MyIndent \MyIndent \MyIndent 
  \triangleright rotw & 0 & 0 \\
  \MyIndent \MyIndent \MyIndent \MyIndent \MyIndent \MyIndent 
  \triangleright sbox & 128 & 0 \\
  \MyIndent \MyIndent \MyIndent \MyIndent \MyIndent \MyIndent 
  \triangleright rcon & 40 & 0 \\

  \MyIndent \MyIndent \MyIndent \MyIndent \triangleright keyblock3 
  & 1854 & 1365 \\

  \MyIndent \MyIndent \MyIndent \triangleright data2state & 0 
  & 0 \\
  \MyIndent \MyIndent \MyIndent \triangleright round & 1535 & 272 \\

  \MyIndent \MyIndent \MyIndent \MyIndent \triangleright subbytes 
  & 512 & 0 \\
  \MyIndent \MyIndent \MyIndent \MyIndent \triangleright shiftrows 
  & 0 & 0 \\
  \MyIndent \MyIndent \MyIndent \MyIndent \triangleright mixcolumns 
  & 176 & 0 \\
  \MyIndent \MyIndent \MyIndent \MyIndent \triangleright addkey & 128 
  & 0 \\

  \MyIndent \MyIndent \MyIndent \triangleright state2data & 1 & 2 \\

  \caption{Hardware usage of entities from synthesis number 3.}
  \label{synt:fifth}
\end{longtable}

%% The plan was to try to reduce the critical path by inserting FFs in 
%% the middle of it and then run another synthesis. This would have 
%% increased the hardware by a lot of FFs, but also increased the 
%% maximum frequency. This was hard to do due to timing issues in the 
%% key generation. Adding some signal delay after keyblock3 would have 
%% reduced the critical path, but also changed a lot of timing 
%% constraints in the keygeneration block FSM. Therefore a decision was 
%% made to change the \emph{User Constraints File} (UCF) instead of 
%% spending the time trying to decrease the critical path, only to 
%% increase the amount of hardware.

%% Running the synthesis with an UCF is supposed to make the synthesis 
%% program optimize the layout of the circuit according to a number of 
%% constraints provided by the user. The constraint made here, was to 
%% reach the desired frequency of 100 MHz. However, no information about
%% whether that frequency could be reached or not was found in the 
%% reports.

%% \subsubsection{Sixth synthesis}
The final place and route yielded the following results:
\begin{verbatim}
Number of Slice Registers: 2,805 out of 126,576    2%
Number of Slice LUTs:      4,930 out of  63,288    7%
\end{verbatim}

\subsubsection{Comparison of synthesis results}
The most interresting numbers from the synthesizes are the number of 
LUTs used, number of ROMs (Read-only Memories) and number of 
Flip-Flops. However, only one ROM has been implemented in this design, 
which was present in all synthesizes, which is why it will not be 
included in the comparison.

\begin{table}[h!]
  \centering
  \begin{tabular}{l | l | l | l | l | l}
    & Second & Third & Fourth & Fifth & Final \\ \hline
    LUTs       & 5121   & 5167  & 5167   & 5167  & 5167  \\ \hline
    Registers  & 4537   & 2945  & 2817   & 2817  & 2818  \\
  \end{tabular}
  \caption{Synthesis results}
  \label{table:synthesis}
\end{table}

%% \begin{figure}[h!]
%%   \begin{equation}
%%     \begin{array}{l | l | l | l | l | l}
%%       & Second & Third & Fourth & Fifth & Final \\ \hline
%%       LUTs       & 5121   & 5167  & 5167   & 5167  & 5167  \\ \hline
%%       Registers  & 4537   & 2945  & 2817   & 2817  & 2818  \\
%%     \end{array}
%%   \end{equation}
%%   \caption{Synthesis results}
%%   \label{table:synthesis}
%% \end{figure}

All of the synthesis results can be viewed in Appendix 
\ref{app:synt_rep}.

There is a difference of one register between the fifth and sixth 
synthesis reports. That register is most likely a residue after an 
attempt to reduce the critical path, and can be disregarded.

\section{Further development}
There are, an amount of optimization that could be performed on the 
circuit. They consist of optimization of code, as well as some deeper 
research into how to rewrite VHDL code to turn the registers in this 
implementation into RAMs, ROMs or LUTs. By rewriting the code into 
those entities instead of the currently implementation, which consists 
of a network of multiplexers, would be a reduction in the critical path
as well as a reduction in the number of used slices.

\subsection{Rijndael's S-Box}
The Rijndael S-box implemented in this design does not synthesize into 
a ROM, which it should be able to do. Other than a ROM, it should also 
be able to be synthesized into a couple of LUT6. It has not been 
possible to find out why the code was implemented into registers 
instead of more efficient solutions.

Both registers and ROMs are viable implementations for the S-box. 
However, for area minimization a ROM might be more favourable, while 
the registers, being clocked, help reduce the critical path of the 
signals.

\subsection{Critical Path}\label{sec:c_path}
To increase the maximum frequency of the circuit, the critical path 
needs to be decreased. This is done by adding FFs in the middle of the 
critical path. This will be hard to solve, due to the complexity of 
the keyexpansion, and would increase the amount of hardware as well as 
the complexity of the circuit if FFs were to be added.

The decision whether to reduce the critical path or not is a hard 
decision due to the amount of hardware that might need to be added to
increase the frequency. However, since LUTs and FFs are located on 
same slice, it is not sure whether the design would use more slices or 
not. A slice is a section of the FPGA, which consists of some basic 
logic.

\section{Tests}
All of the entities in the design have been simulated and evaluated 
seperately before being merged and tested together, to make sure that 
they have the desired functionality both seperately and when combined 
together. The simulations of the seperate blocks are trivial, and 
therefore not included in the report. 

Figure \ref{test:1} through \ref{test:3} are tests performed on the 
complete AES128 block, before CBC-mode. In the figures, in\_key is 
the input key to be extended and used, and datapacket is one packet 
from a TS. Test vector 1 and 2 are taken from \citet{AES:2001}, while 
test vector 3 is generated using webpage \citep{Generator}.

\emph{Test vector 1 (Figure \ref{test:1})}\\
Input key: 2b 7e 15 16 28 ae d2 a6 ab f7 15 88 09 cf 4f 3c\\
Plaintext: 32 43 f6 a8 88 5a 30 8d 31 31 98 a2 e0 37 07 34\\
Ciphertext: 39 25 84 1d 02 dc 09 fb dc 11 85 97 19 6a 0b 32

\begin{figure}
  \includegraphics[width=\textwidth]{successtv1}
  \caption{Test vector 1}
  \label{test:1}
\end{figure}

\emph{Test vector 2 (Figure \ref{test:2})}\\
Input key: 00 01 02 03 04 05 06 07 08 09 0a 0b 0c 0d 0e 0f\\
Plaintext: 00 11 22 33 44 55 66 77 88 99 aa bb cc dd ee ff\\
Ciphertext: 69 c4 e0 d8 6a 7b 04 30 d8 cd b7 80 70 b4 c5 5a

\begin{figure}
  \includegraphics[width=\textwidth]{successtv2}
  \caption{Test vector 2}
  \label{test:2}
\end{figure}

\emph{Test vector 3 (Figure \ref{test:3})} \\
Input key: 10 20 30 40 50 60 70 80 90 a0 b0 c0 d0 e0 f0 bb\\
Plaintext: 00 11 22 33 44 55 66 77 88 99 aa bb cc dd ee ff\\
Ciphertext: bf 99 1f aa 8b 0f e6 48 36 46 a0 2d 33 9e de a5

\begin{figure}
  \includegraphics[width=\textwidth]{successtv3}
  \caption{Test vector 3}
  \label{test:3}
\end{figure}

\section{Comparison to other implementations}
Since there exist more implementations of the AES128 algorithm, a 
comparison between the implementations is interresting. To be noted is 
that this implementation has been focusing on minimizing the hardware 
usage, while achieving a throughput of atleast 1 Gbit/s. Since it is 
possible to achieve an entirely combinatorial AES128 scrambler, as 
well as a pipelined version, the focus has been to try to find an 
implementation as similar as this one as possible to compare the 
results.

Note that this implementation has got a couple of entities, that are 
usually not present in scramblers. They are there in order to allow 
insertion of the implementation into the FPGA used by WISI.

The design implemented in this thesis scrambles a block of 16 bytes 
of data in 13 clock pulses. It uses 4930 LUTs, occupies 1727 slices 
and can be run at a maximum frequency of 94 MHz. The circuit is 
implemented on a Xilinx Spartan-6. This can be compared to the Fast 
AES XTS/CBC implementation (Helion) made by Xilinx. It uses 1041 
slices and 4047 LUTs with a maximum frequency of 130 MHz. 
\cite{Xilinx:AES} 

The implementation by Xilinx uses custom FPGA optimization techniques 
through hand crafted macros. A comparison between the two 
implementations can be found in table \ref{fig:imp_diff}. When these 
two implementations are compared, the most important difference is the 
frequency at which the two circuits can be run. The amount of LUTs 
that are used does in fact not mean much, since a LUT is said to be 
used even if just a small part is used in the circuit. If only one LUT 
in a slice is used, the LUT is still considered used. 

The percentage row in table \ref{fig:imp_diff} indicates how many more 
percent LUTs and slices that are used in this implementation compared 
to the Xilinx implementation. However, the frequency shows how much 
higher the frequency of the Xilinx implementation is.

\begin{table}[h!]
  \centering
  \begin{array}{| l | l | l | l |}
    \hline
    & Slices & LUTs & Frequency (MHz) \\ \hline
    Current & 1727 & 4930 & 94 \\ \hline
    Xilinx & 1041 & 4047 & 130 \\ \hline
    Percentage & 65\% & 21.8\% & 27.7\% \\ \hline
  \end{array}
  \caption{Comparison between implementations.}
  \label{fig:imp_diff}
\end{table}

%% The current implementation, compared to Xilinx implementation has got
%% a lower maximum frequency, while using more space on the FPGA. This is
%% however a good result, given that Xilinx solution is commercial, and 
%% implemented on a Xilinx board by Xilinx.

\section{Discussion}
One of the first things noticed during this thesis was that industrial
secrecy can put a quick halt to projects. A license had to be written 
and approved by ETSI before WISI Norden was allowed information about 
the specifications of one of the algorithms that were supposed to be
analyzed. Due to restrictions, this could not be done, meaning that 
this specific analyzis came to a halt before it even started. This 
led to the comparison between a software- and hardware-friendly 
algorithm becoming impossible to do. 

While WISI Norden and ETSI were discussing the license, specifics 
about the CSA3 and CISSa algorithms were investigated, since those 
were the ones to be analyzed. From the little information available 
about the CSA3, only the names of the two ciphers were possible to 
find. Since one of the two ciphers in the CSA3 algorithm corresponded 
to one of the ciphers in the CISSA algorithm, this cipher was examined 
as much as possible. This was the AES128 cipher.

Both the key generation and the functionality of the entities in 
AES-128 algorithm could be found in litterature, since the AES 
encryption is a public algorithm. From what could be found out about 
the CISSA algorithm, through an official ETSI journal, it seemed to 
just use the AES-128 algorithm, but in a certain mode, with a specific 
Initialization Vector.

%% \section{Discussion}
%% The main reason for choosing a bottoms-up methodology, was since the 
%% functionality of the smaller blocks were very basic, but the timing 
%% was rather complex. In addition to the fact that there were no 
%% concrete guidlines, performing the work on the smaller entities, and 
%% then implementing the higher ones minimized the probability of 
%% performing tasks that were to be discarded in later stages of the 
%% implementation.

\section{Conclusions}
An analysis of two algorithms proposed to replace the current standard 
scrambling algorithm CSA has been done. This has been done through 
a study of litterature and documents defining the algorithms. The 
system was built using a bottom-up design, after finding basic 
information about the components of the algorithm.

As explained in chapter \ref{csa3cissa}, the lack of information 
available about the two algorithms made an analysis of what made one 
of them hardware friendly and the other software friendly impossible. 
However, CSI+ requires the scrambling to be done using the AES128 
algorithm in CBC-mode, and therefore CISSA was chosen to be 
implemented.

As mentioned in chapter \ref{ssec:hu}, optimization of the circuit has 
been mainly done by minimizing the usage of registers. This has been 
done by increasing the amount of signals. Most decisions were made 
through analysis of simulations and synthesis reports.

The biggest difficulties encountered during this thesis has been 
finding proper documentation. Moreover, as mentioned in 
\ref{ssec:merge}, merging has taken more time and focus than was 
planned for.
