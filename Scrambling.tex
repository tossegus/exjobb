\chapter{Cryptography}\label{ch:Scrambling}
Cryptography is the science of rendering content incomprehensible for 
undesired readers. Non-encrypted content is called plaintexts, and 
encrypted plaintexts are called ciphertexts in cryptography. However, 
securing content is not only about cryptography. The main reason why 
the encryption algorithms are attacked, is because an attack has a 
very low chance of being detected. There will be no traces of the 
attack, since the attacker’s access will look just like an ordinary 
access. \citep{Schneier:2003}

This can be compared to a real-life break-in. The break-in will be 
noticed if the thief breaks in using a crowbar. On the other hand, you 
might never notice that the security had been breached, if the the 
thief were to pick the lock instead. \citep{Schneier:2003}

One of the more important cryptography rules, is to always assume that 
someone is out to get you. Because of this, 
\citet[pp. 12--14]{Schneier:2003} claims that it is always important to
look for possible ways of breaking systems. Possible ways of bypassing 
the encryption and ways of breaking algorithms are more easily noted 
by looking at systems with this mindset, and doing so allows 
cryptographers to more easily notice faults, which can then be fixed 
and thereby provide a more robust security system.

\section{Why cryptography is needed}
Cryptography is the science of rendering plaintexts into ciphertexts to
protect contents from unauthorized viewing. It is used in electronic 
communication for protection of e-mail messages and credit card 
information among other things. If data is sent without being 
encrypted, someone listening in to the transmission channel will be 
able to access the data.

For most people this is not a problem, but in some instances sending 
secure messages can be extremely important. One example is 
communication during war, where a single piece of intelligence might 
turn the tide of the entire war. Moreover, you do not want people to 
read your account information or credit card number when you do online 
shopping. Another reason is to make sure that users do not access 
premium content without paying for it, as the case is with DVB. All of 
these problems can be solved by cryptography and encryption.

%%%%%%%%%%%%%%%%%%%%%%%%%%%%%%%%%%%%%%%%%%%%5
%\Warning[Source]{Typ Schneier har jag för mig? Låna om boken?}
%This might need a rework!
%%%%%%%%%%%%%%%%%%%%%%%%%%%% %%

%% \section{Scrambling or Encrypting}
%% Scrambling can be seen as the distortion of a plaintext, using a 
%% control word. However, encryption can be seen as the entire process of 
%% protecting content. This includes everything from generating the 
%% control word to scrambling data. Scrambling can therefore be seen as a 
%% part of encryption.

%% %Jag vet inte om det här är relevant, eller stämmer

%% Yet another reason to scramble, is to reduce the number of adjacent 
%% data bits with the same value, like strings of zeroes or ones. It could
%% also serve to balance the number of zeroes and ones in strings. This 
%% is done as to try to obtain DC balance. DC balance is desired since it 
%% avoids voltage imbalance during communication between connected 
%% systems.
%% %%%%%%%%%%%%%%%%%%%%%%%%%%
%% %\Warning[Rewrite]{And find a good source for this}
%% %Det är osannolikt att det här stämmer helt och hållet :)
%% %%%%%%%%%%%%%%%%%%%%%%%%%%
\section{Data packets}\label{sec:Data}
The data processed by the DVB systems is packaged into data packets 
before it is sent. All of them are created from ES packets 
(Elementary Stream) which are generally the output from an audio or 
video encoder. The ES-packets are packeted into PS- (Program Stream), 
TS- (Transport Stream) or PES packets (Packatized Elementary Stream) 
before being distributed. Among the three ways of packing data, only 
two are interresting from a DVB perspective. This is due to PS packets 
being used for storing data, while TS and PES are used for 
transmitting data. The interresting types, when working with DVB, are 
therefore the TS packets as well as the PES packets. PES packets are 
often packed into the payload of TS packets. TS packets are desirable 
since they are of a fixed length of 188 bytes, while the PES packets 
are desirable due to their strength. The payload is the part of the 
packets which is the actual data, which is everything except the 
header and adaptation field.

\subsection{TS packets}
TS packets are used by the DVB society due to their fixed length, and 
the fact that TS packets are meant to be used for streaming services, 
while PS packets are used for storing packets of data. TS packets have 
got a length of 188 bytes with a 4 byte long header. This means that 
the payload consists of a maximum of 184 bytes. The layout of a TS 
packet can be viewed in figure. \ref{img:Package}\citep{DVB:2013}

\begin{figure}[h!]
  \includegraphics[width=\textwidth]{TSPacket2}
  \caption{General layout of a data packet}
  \label{img:Package}
\end{figure}

The TS packet consists of 4 different kinds of building blocks where 
only the header is guaranteed to be present. Those blocks are:

\begin{itemize}
\item Header
\item Adaptation field
\item Encrypted payload
\item Clear payload
\end{itemize}

The byte-sizes of the building blocks of a TS packet are given in 
table \ref{table:sizes}. The block\_size is the size of the blocks 
that are encrypted, and is 16 for all of the AES standards.

\begin{longtable}{| l | l |}
  \hline
  Part & Size in bytes \\ \hline
  header\_size & 4 \\ \hline
  adaptation\_field\_size & the size of the adaptation field \\ \hline
  payload\_size & 188 - (header\_size + adaptation\_field\_size) \\ 
  \hline
  encrypted\_payload\_size & payload\_size - [payload\_size mod 
    block\_size] \\ \hline
  clear\_payload\_size & [payload\_size mod block\_size] \\
   & (or simply payload\_size - encrypted\_payload\_size) \\ \hline 
  \caption{Byte sizes of the parts in TS packets. The TS packet is 
    188 bytes.}
  \label{table:sizes}
\end{longtable}

The header is always 4 bytes, while the adaptation field can have any 
size between 0 and 183 bytes. This means that the clear payload can 
be of any size stretching from 0 bytes, to one byte smaller than the 
block size. The rest of the data consists of the payload.

\subsubsection{Header}
The header consists of information regarding the packet, and has a 
sync\_byte (with a hex-value of 0x47, or bit-value of 01000111) to 
announce the beginning of a packet. The value of the sync\_byte 
corresponds to the ASCII-value of the letter G which stands for Go.
The header also contains information as to whether there is an 
adaptation field and payload in the packet, what Packet ID (PID) the 
packet has, if it should be prioritized, whether the data is 
scrambled - and in that case if it was scrambled with an odd or even 
key, among others \citep[pp. 25--26]{etsiMPEG:2009}. The header should 
never be encrypted and is always found at the beginning of a packet 
\citep[pp. 10--11]{DVB:2013}.

The header contains the following bits:
\begin{longtable}{| l | l | l |}
  \hline
  Bits & Name & Description \\ \hline
  8 & Sync byte & Fixed byte value 0x47 \\ \hline
  1 & Transport Error Indicator & Uncorrectable bit errors exist \\ 
  \hline
  1 & Payload Unit Start Indicator & TS packet contains PES packets or\\
  & & Program Specific Information (PSI data)\\ \hline
  2 & Transport Scrambling Control & 00 No scramling, 01 Reserved, \\
  & & 10 Even key, 11 Odd key \\ \hline
  1 & Transport Priority & 1 gives this packet higher priority \\ 
  \hline
  13 & PID & Packet identification number \\ \hline
  1 & Adaptation Field Control & Adaptation field exists \\ \hline
  1 & Contains Payload & Payload exists \\ \hline
  4 & Continuity Counter & Packet number. Used to make sure \\
  & & packets are not lost\\ \hline
\end{longtable}

\subsubsection{Adaptation field}
The adaptation field is a padding field that is only inserted when the 
end of the data does not align with the end of the TS packet. This is 
done to make sure that the TS packet is filled with known data. 
Adaptation fields are never encrypted. \cite[pp. 10--11]{DVB:2013}

\subsubsection{Encrypted and clear payload}
Clear bytes of data tend to turn up, when working with block ciphers. 
This happens since block ciphers only encrypt data blocks of fixed 
sizes. The clear data is always located at the end of the received 
TS packet. When receiving a TS packet, the first thing to be done is 
to find the start of the payload. The start of the payload is found 
directly after the header, when there is no adaptation field. If an 
adaptation field is present, we can find the data after it. The length 
of the adaptation field is found in the beginning of it. When the 
start of the payload has been found, blocks of a given size are sent 
to the scrambler. The remainder of the data, when all of the blocks 
of the right size have been scrambled, is to be left clear. The number 
of unscrambled bytes might be of sizes up to one byte smaller than the 
block size. This means that the AES-128, which works on block sizes of 
16 bytes, can have a maximum of 15 clear bytes. 
\cite[pp. 10--11]{DVB:2013}

\subsection{PES packets}
The PES packets have varying lengths of up to 64 kilo bytes, and are 
often packed into TS packets when distributed, due to the TS packets 
being strong. The payload data in the TS packets, when carrying PES 
packets, consist of the entire PES packets, which is the header as 
well as the data. PES packets do not use adaptation fields, since 
they are of adaptable lengths, as long as the length of the packet 
does not exceed 64 kilo bytes.

Since Digital Video Broadcasting seldom uses PES packets, an 
explanation of the elements of the PES header will not be done in 
this report. The derivation of PES packets from TS packets can be 
seen in Figure \ref{img:PES} \citep[p. 9]{ETR:289}.

\begin{figure}[h!]
  \includegraphics[width=\textwidth]{PES.png}
  \caption{PES packet derived from TS packets. The packet in the top 
    is a PES packet and the packets in the bottom are TS packets.}
  \label{img:PES}
\end{figure}

\section{Encryption and Decryption}
There are three things that you need when you encrypt and decrypt 
messages. Those are the algorithm, plaintext and the key. Even though 
there are plenty of ways to encrypt messages, there are mainly two ways
of sharing the encryption-key. The first method is the symmetric-key 
encryption, and the second method is the public-key encryption. 
\citep{Schneier:2003}

\subsection{Symmetric-key encryption}\label{ch:symmetric}
The symmetric-key encryption uses the same key to encode and decode 
messages. Distrubution of the key, when using the symmetric-key 
encryption is troublesome and the fact that both parties need access 
to the same secret key is a major drawback of the symmetric key 
encryption, as compared to the public-key encryption method. Sending 
the key in an email is a bad idea, since the persons who want to read 
the sent messages are most likely already listening. They will 
therefore obtain the key as well as the means to decode the messages. 
Both the CSA and the AES encryption methods are symmetric-key 
encryptions, using the same key for encryption and decryption. 
\citep{Schneier:2003}

\subsection{Public-key encryption}\label{ch:public}
The public-key encryption uses a public key that anyone can look up, 
and a secret key that only one person knows 
\citep[pp. 25--32]{Simmons:1992}. For instance say that two persons, 
Bob and Alice, want to communicate. Bob produces a keypair \(P_{Bob}\) 
(Bob’s public key) and \(S_{Bob}\) (Bob’s secret key) and publishes 
\(P_{Bob}\) for anyone to see. When Alice wants to send Bob a message, 
she looks up Bob’s public key \(P_{Bob}\), which she uses to encode her 
message. When she sends Bob the message, Bob decodes the message using 
his secret key \(S_{Bob}\). Since Alice now knows both 
the plaintext, and can find out what the corresponding ciphertext will 
be, she could potentially try to find Bob's secret key. 
\citep{Schneier:2003}

\subsection{Combination of encryption methods}
If the public-key encryption seems secure and easy to manage, how 
come it is not the only encryption method used? The reason is that the 
public-key encryption is not as effective as the symmetric-key 
encryption. It is common to use a combination of those two when 
an easy, effective way to encrypt messages is desired.

To combine the two encryption methods, the symmetric-key algorithm 
encodes the plaintext into a ciphertext. Then the public-key 
encryption encrypts the key used by the symmeteric-key encryption. The 
encoded key is then sent together with the ciphertext to the 
recipient, which decodes the symmetric key using the secret key. The 
plaintext is obtained by decrypting the ciphertext using the symmetric 
key.

%% HERE DO THIS PART NOW!! YOU GOT ALL THE WAY HERE!!

\section{Ciphers}
A cipher is the same as an encryption algorithm, which operates on 
either plaintexts or ciphertexts to perform encryption or decryption. 
Figure \ref{img:ciphers} shows how the different kinds of ciphers 
can be split into sub-groups. The first branch splits into Classical-, 
Rotor Machine- and Modern ciphers. Substitution and Transposition are 
still used in modern algorithms, even though they are considered 
classical ciphers. The Modern ciphers are the Private 
key and Public key (descibed in chapter \ref{ch:symmetric} and 
\ref{ch:public}). The CSA algorithm uses both the stream- and block 
ciphers, while the AES algorithm only uses a block cipher.

\begin{figure}
  \includegraphics[width=\textwidth]{cipher.png}
  \caption{Different kinds of ciphers. \citep{CipherTax:2013}}
  \label{img:ciphers}
\end{figure}

There are mainly two kinds of ciphers that are used when designing 
modern cryptosystems. Those ciphers are called block ciphers and 
stream ciphers. Many systems use a combination of block ciphers and 
stream ciphers to provide security. 
%%%%%%%%%%%%%%%%%%%%%%%
%\Warning[Source]{At least the CSA uses it. But you need a source}
%%%%%%%%%%%%%%%%%%%%%%%

\subsection{Block cipher}\label{sec:BlockCipher}
A block cipher operates on fixed sized sets of data. These sets are 
called blocks, which is the reason why they are called block ciphers. 
Them being fixed sizes might cause a need for padding of the blocks, in 
case the plaintext contains a number of bytes that is not a multiplier
of the blocksize. Block ciphers often use a combination of Substitution-boxes 
(S-boxes) and Permutation-boxes (P-boxes) in a so-called SP-network 
(S-box / P-box network) (Figure \ref{img:SPNetwork}). There are many modes 
of block ciphers, but the two recommended by \citet{Schneier:2003} are the 
CBC-mode and the CTR-mode, which are described in the following sections.

\subsubsection{CBC} \label{sec:cbc}
CBC stands for \emph{cipher block chaining} and is performed by 
encrypting the result of an XOR (basic logic component) between 
an Initialization Vector (IV) and the plaintext. The resulting 
ciphertext is then fed back to the XOR replacing the IV.
This means that the data input into the cipher will be the result of 
an XOR between the previous result, and the next plaintext. This is 
then put into an XOR with the next plaintext, which is then encrypted 
in the cipher. For reference, see image \ref{img:CBCmode}. 
\citep[pp. 109--111]{Stinson:2006}

\begin{figure}
  \begin{center}
   \includegraphics[width=\textwidth]{CBCmode}
  \end{center}
  \caption{Cipher block chaining mode, \citep{CBCmode:2014}}
  \label{img:CBCmode}
\end{figure}

\subsubsection{CTR}
CTR stands for \emph{counter}, and refers to the way the IV is 
generated. The counter outputs a value, which is encoded with the key. 
The counter often uses a Linear Feedback Shift Register (LFSR) with some 
sort of logical function, most often XORs. The counter has got an internal
state, which it manipulates in order to create the next state. The state
is what is output from the counter and sent to an XOR together with the 
plaintext, producing the ciphertext. The counter is incremented and the 
procedure is iterated \citep[p. 111]{Stinson:2006}.

\subsection{Stream cipher} \label{sec:StreamCipher}
Stream ciphers work on streams of data. They usually consist of a 
keystream generator which performs an XOR with the data 
\cite[pp. 67]{Simmons:1992}. An effective implementation of the stream 
cipher is to use a linear feedback shift-register which uses the 
current internal state (key) to produce the next state by a simple 
XOR-addition between two or more of the bits in the state. This is 
mainly used because of how easy it is to construct in hardware 
\citep{LFSR:2008}.

\subsection{Decryption}
Decryption is often performed by reversing the encryption. You need to 
know the algorithm, preferably through a mathematical representation, 
to calculate how to obtain the plaintext from the ciphertext. The 
inverse function of the basic building blocks also needs to be known.

%% An 
%% example of how this is done for the CBC-mode (described in 
%% \ref{sec:BlockCipher}) is described in \ref{sec:CBCcalc} in appendix 
%% \ref{app:examples}. It is here assumed that the decryption algorithm 
%% is known, for simplicity.
%%%%%%%%%%%%%%%%%%%%%%%
%\Warning[Source]{feels like a given, but still}
%%%%%%%%%%%%%%%%%%%%%%%

\section{Confusion and Diffusion}\label{ch:ConfDiff}
Two properties that are needed to ensure that a cipher provides 
security are confusion and diffusion \citep{Shannon:1949}. 
\emph{Confusion} refers to making the relationship between ciphertext 
and key as complex as possible. \emph{Diffusion} refers to replacing 
and shuffling the data, to make it impossible to analyze data 
statistically. This is usually done by performing substitutions and 
permutations in a simple pattern multiple times. This can easily be 
done by using an SP-network \citep[pp. 74--79]{Stinson:2006}. The first
 as well as the last step of SP-Networks is usually an XOR between 
the subkey and the data. A subkey is a key generated from the provided 
key, and is used to provide systems with more complex encryptions. 
Performing an XOR between a subkey and data is called \emph{whitening}, 
and is according to \citet[p. 75]{Stinson:2006} regarded as a very 
effective way to prevent encryption/decryption without the key. 
The goal of this is to make it hard to find the key, even though one 
has access to multiple plaintext/ciphertext pairs produced with the 
same key \citep{Shannon:1949}.

However, a cipher is not guaranteed to be secure just because it 
provides these two properties.

\subsection{S-boxes and P-boxes}
The S-box is one of the basic components that is used when creating 
ciphers. An S-box takes a number of input bits and creates an equal 
number of output bits. The way they are generated is non-linear. 
Implementing an S-box can effectively be done using lookup tables, since
the function of an S-box is to substitute the input with a different 
output, which corresponds to the functionality of a lookup table.
Each input has to correspond to a unique output, to make sure that the 
functionality of the S-Box can be uniquely reversed. If it can not be 
reversed, descrambling will be impossible. 
\citep[pp. 74--75]{Stinson:2006}

% CONFUSION OR DIFFUSION?
The second basic component used in cryptography is the P-box. A P-box 
shuffles and thereby rearranges the order of given bits. This can be 
viewed in the SP-network in figure \ref{img:SPNetwork}, where the P-box 
is represented by the dotted rectangle in the middle.

\begin{figure}[h!]
  \begin{center}
    \includegraphics[width=0.5\textwidth]{SPnetwork}
    \caption{SP-Network}
    \label{img:SPNetwork}
  \end{center}
\end{figure}

%CONFUSION OR DIFFUSION?

\section{Secrecy}
Although encryption is important, as well as the strength of the 
encryption, neither using an algorithm designed for use in just a few 
applications nor using a secret algorithm is ever a good idea. A simple 
mistake when designing an algorithm might turn an encryption that would 
otherwise have been strong, incredibly weak. If you use an open 
algorithm, faults will most likely be discovered and fixed by 
experienced cryptographers \citep[pp. 23]{Schneier:2003}. Keeping the 
key, which is used to encrypt the data, secret is what is important.
