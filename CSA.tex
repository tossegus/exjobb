\chapter{Common Scrambling Algorithm} \label{ch:CSA}
The Common Scrambling Algorithm (CSA) is currently the most commonly 
used encryption algorithm for encryption of video-streams in the DVB 
context. The CSA uses a combination of a block cipher, taking an input 
of 64-bit blocks, and a stream cipher. Both of the ciphers use the 
same key, which means that the entire system uses the same key. This 
means that the complete algorithm would break if the key would be 
recovered, as long as the person recovering the key knows what the 
decryption algorithm looks like. On the other had, using the same key 
for the whole system allows for easily changing keys at regular 
intervals. \citep[pp. 271--272]{WeiLi:2007}

CSA has been the official scrambling method for protecting DVB content 
since may 1994. CSA was to be easily implemented in hardware and hard 
to implement in software to, among other reasons, make 
reverse-engineering of the algorithm difficult \citep{DVBScene:2013}.

There are two versions of the DVB-CSA, CSA1 and CSA2, where the 
key-length is the only difference.
\citep[p. 23]{DVBScene:2013}

\section{The need for a new standard}
The DVB-CSA standard offers short-term protection, while it assumes 
content is viewed in real time and not stored. Due to the development 
of how content has come to be consumed during recent years, the focus 
has shifted from transmitting contents to primarily being able to 
distribute content across homes. As a result of this, functionality 
needs to be changed from securing the delivery of content, to securing 
the content. \citep{Farncombe}

Another thing to bear in mind is the fact that more CPU-based units, 
such as smart-phones, tablets and computers are used to access contents 
now more than ever. In order to allow for descrambling on CPU-based 
units, a software-friendly scrambling algorithm is required.

\section{Layout of the CSA}
The CSA consists of a block cipher and a stream cipher which are 
connected in sequence \citep[p. 271]{WeiLi:2007}. The block cipher 
reads blocks of data, each consisting of 64-bits, which are run in 
CBC-mode (see section \ref{sec:BlockCipher}). The block cipher 
processes these blocks of data in 56 rounds. The output of the block 
cipher is sent to the stream cipher, where additional encoding is 
performed. The first block of data sent from the block cipher to the 
stream cipher is used as an IV for the stream cipher. It is therefore
not encoded the stream cipher. \citep{DVBAnalysis:2006}

\section{Security}
One of the problems associated with control word distribution is that 
control word sharing has become rather common \citep{Farncombe}. 
Control word sharing is primarily done by connecting several set-top boxes
into a network in order to share the decrypted control word, and get access
to more content. It has probably become more popular since the control 
words are sent in the clear between the smart card and the STB, meaning 
that a user might grab the clear control word during transmission and 
redistribute it over the internet. This has become a financial problem 
for content distributors, since people have stopped paying for the 
content that they are watching.

One way of dealing with control word sharing is to decode the encrypted 
control word on the CI system. The control word is then encrypted once 
again on the CI before it is transmitted to the STB. The key used for 
the second encryption is setup between the CI system and the STB through 
a one time sychronization. This means that users are not able to grab 
the clear control word and redistribute it. \citep[pp. 12--13]{HIS:2011}

Another security issue that you need to think of when designing the 
hardware, is to make sure that no contacts are ever accessible from 
the top layer of the circuit board. This is due to the fact that 
people would be able to connect hardware to the board and download 
the material that way, if they were.
%%%%%%%%%%%%%%%%%%%%
%\Warning[Source]{Except from Patrik Lantto}
%%%%%%%%%%%%%%%%%%%%
There also exists a need to be aware of people trying to break the 
algorithm through forced ways, as well as control word sharing and 
hardware methods of stealing content.

\subsection{Breaking the CSA}
There are a few standard ways to try to break ciphers. The most common 
ones are the brute force-, known plaintext-, chosen plaintext- and 
birthday attacks. You choose what method to use depending on what the 
design of the cipher is. The most relevant ones, in the context of the 
CSA, will be explained in the following subsections.
\citep[pp. 31-34]{Schneier:2003}

\subsubsection{Brute force}
The number of unique keys that can be extracted depends solely on the 
length of the key. The number of combinations corresponds to the 
largest number, plus one. The formula for the largest possible number 
obtainable, when working on keys represented as binary numbers, where 
the key-length is represented by the letter n, can be viewed in equation
\ref{eq:num}. Note that the key-length is given in number of bits.

\begin{equation} 
  \text{Largest possible number} = 2^{n} - 1
  \label{eq:num}
\end{equation}

Since the CSA uses keys consisting of 64-bits (8 bytes), this gives us 
18.5\times$10^{18}$ possible keys. However, \emph{byte} 3 and 
7 are often used as parity bytes in CA systems, which leads to only 48 
bits being used in the key.  This can be seen in figure 
\ref{fig:blockcipher}. 48 bits on other hand would lead to $2^{48}$ 
combinations, which corresponds to 281\times$10^{12}$ possible 
unique keys. 

Testing a million keys per second is about what is possible through a 
modern x86 processor using software methods. This means it would take 
roughly 3258 days to force break the keys, which translates into 
roughly 8.8 years. The calculations are done in equation \ref{eq:keys} 
through \ref{eq:days}. \citep{Breaking:2012}

Number of unique keys, for a 48-bit key:
\begin{equation}
  2^{48} = 2,8147497\times10^{14} \text{ keys}
  \label{eq:keys}
\end{equation}

By dividing by the number of tested keys per second, the number of 
seconds to test all the keys will be found:
\begin{equation}
  281\times10^{12} / 10^{6} = 281,4749767\times10^{6} \text{ seconds}
  \label{eq:seconds}
\end{equation}

By substituting seconds with days $\times$ (seconds/day), the number of 
keys per days is found instead:
\begin{equation}
  281,4749767\times10^{6} / 86400 = 3257,8 \text{ days}
  \label{eq:days}
\end{equation}

\begin{figure}[h!]
  \begin{center}
    \includegraphics{blockcipher}
  \end{center}
  \caption{Number of bits in key used.}
  \label{fig:blockcipher}
\end{figure}

Moreover, systems need to change the key at least every 120 seconds 
\citep{Simpson:2009}. Changing the key every second minute would mean 
that 140 trillion keys would need to be scanned per minute, to cover 
the most of the keys in the two minutes available before the key is 
changed. However, most systems issues new keys between every 10th - 
120th second, which means that for some systems 28.1 trillion keys need 
to be scanned per second \citep{Wirt:2004}.

It is possible to use dedicated hardware and FPGA implementations to 
speed up the force breaking methods through hardware acceleration. This 
could make it possible to scan through 2.8 trillion keys per second, 
just barely allowing us to be certain to find the key in two minutes. 
Even so, the key could be changed more frequenctly than every second 
minute. As such, the brute force method of is not a reasonable method 
to obtain the key. 

\subsubsection{Known plaintext attack}\label{sec:kpa}
Known plaintext attacks are performed to figure out the key. What is 
interresting is that this kind of attack only is applicable for 
symmetric ciphers. That means that the known plaintext attack cannot 
be used to retrieve the secret key during public-key encryption. The 
key can then be used to decrypt following ciphertexts. To perform this 
kind of attack, a known plaintext-ciphertext pair is needed. You can 
try to find the key if you have the both of them. This is done by 
identifying ciphertexts known to correspond to zero-filled plaintexts, 
when trying to break the CSA \citep{Breaking:2012}. Memories are then 
filled with precalculated keys, which are used to find which key the 
current plaintext-ciphertext pair corresponds to. This method is 
supposed to recover a key in roughly 7 seconds with a 97\% certainty 
according to \citet{Breaking:2012}.
