\chapter{CSA} \label{ch:CSA}
The CSA is currently the most commonly used encryption algorithm in DVB for 
encryption of video-streams. There are two versions of the DVB-CSA, CSA1 and 
CSA2. The only difference between them being the key-length 
\citep[p. 23]{DVBScene:2013}. The CSA uses a combination of a block cipher 
taking an input of a 64-bit block, and a stream cipher. Both of the ciphers use 
the same key, so that the entire system uses the same key 
\citep[pp. 271--272]{WeiLi:2007}. This means that the complete algorithm would 
break if the key would be recovered. Using the same key does on the other hand 
allow us to change the key at regular intervals more easily. 

CSA has been the official scrambling method for DVB since may 1994. CSA was 
to be implemented in hardware and hard to implement in software, for the sake
 of making reverse-engineering difficult \citep{DVBScene:2013}.

\section{History} \label{sec:History}
CSA was largely kept secret until 2002. It was made very hard to reverse engineer
since it was largely implemented in hardware. The patent papers gave some hints
of the layout, but important details like the layout of the S-boxes remained 
secret.
Without the S-boxes, free implementations of the algorithm were out of question. 
Initially, CSA was to remain implemented in hardware only, but software 
implementations were found on the internet, which made it possible to analyze
the entire solution.

In 2002 FreeDec was released, implementing CSA in software. Though released as 
binary only, disassembly revealed the missing details and allowed 
reimplementation of the algorithm in higher-level programming languages. 
\Warning[source]{Is this relevant, and you need another source than Wikipedia}

With CSA now publicly known in its entirety, cryptanalysts started looking for 
weaknesses.

\begin{figure}
  \begin{center}
    \includegraphics{blockcipher}
  \end{center}
  \caption{Image from \citet[pp. 49]{Breaking:2012}}
  \label{fig:blockcipher}
\end{figure}

\section{Layout of the CSA}
The CSA consists a block cipher and a stream cipher connected in sequence 
\citep[p. 271]{WeiLi:2007}. The block cipher reads 64-bit blocks of data, which 
is then run in an CBC-mode. The block cipher processes these blocks of data in 
56 rounds. The output of this is sent to the stream cipher where additional 
encoding is performed. The first block of data is used as an IV for the stream 
cipher, and is not encoded in this phase. \citep{DVBAnalysis:2006}

\section{Breaking the CSA}
There are a few standard ways to try when you want to break a cipher. 
Those are the brute force approach, known-plaintext attacks, chosen plaintext 
attacks and birthday attacks \citep[pp. 31-34]{Schneier:2003}. You choose what 
method to use depending on what the ciphers look like. I will not discuss all 
of them, but I will talk about the most relevant ones here.

\subsection{CW sharing}
One of the problems with the CW distribution is the fact that CW sharing has 
become a real problem \citep{Farncombe}. This is possible due to the fact that
the CW is sent in the clear between the smart card and the STB, meaning that a 
user might grab the clear CW and redistribute it over the internet. 
This has become a huge problem 
One way is to decode the encrypted CW on the smart card, and then encrypt it 
once again on the smart card. The latter key has been setup between the smart 
card and the STB, and is done through a sychronization one time. This means that 
users are not able to grab the clear CW and redistribute it. 
\citep[pp. 12--13]{HIS:2011} 

\subsection{Brute force}
The CSA uses a key consisting of 64-bits, which gives us 18.5 Quintillion 
possible keys (Quintillion is $10^{18}$). But byte 3 and 7 are often used as 
parity bytes in CA systems which leads to only 48 bits being used in the key 
\citep{Breaking:2012}. This can be seen in figure \ref{fig:blockcipher}. 48 bits 
on other hand leads to $2^{48}$ combinations, which corresponds to 281 trillion 
possible keys (Trillion is $10^{12}$). Testing a million keys per second is about 
what is possible through on a modern x86 processor using software methods
\Warning[Todo]{How did I get this number?}, which means it would take roughly 
3258 days to force brake the keys. That is roughly 8.8 years.

Moreover, systems need to change the key at least every 120 seconds \citep{Simpson:2009} and most systems issues new keys every 10-120 second \citep{Wirt:2004}.

It is possible to use dedicated hardware and FPGA implementations to speed this 
up, using hardware accelerations and other methods. But even if we would then 
be able to scan through 2.8 trillion keys per second, precisely allowing us to 
be certain to find the key in two minutes, we could just change the key more 
often. As such, the brute force method of obtaining the key is not an option.

%\subsection{Bit slicing}
%The stream cipher part of CSA is prone to bit slicing, a software implementation 
%technique that allows decryption of many blocks, or the same block with many 
%different keys, at the same time. This significantly speeds up a brute force 
%search implemented in software, although the factor is too low to make a real-time 
%attack practical.

%The block cipher part is harder to bit slice, as the S-boxes involved are too large 
%(8x8) to be efficiently implemented using logical operations, a prerequisite for bit 
%slicing to be more efficient than a regular implementation.

\section{Why do we need a new standard?}
The DVB-CSA standard offers short-term protection (it assumed content is viewed 
in real time and not stored). But due to the development during recent years, 
we now need to be able to distribute the content around the home. This means 
that the focus needs to be moved from securing the delivery, to securing the 
content. \citep{Farncombe}


There are currently two implementations suggested for doing this. The first is 
the hardware friendly CSA3 which uses itself of AES blocks (with keys of sized 
128, 192 or 256 bits) as well as a confidential block cipher called the XRC. The 
second version is the software friendly CISSA which uses the AES-128 as a basic 
building block. \citep{DVB:2013}

What are the pros and cons of having hardware scramblers and software scramblers.
Software uses itself of the date to generate a random key. Hardware uses itself 
of additive scramblers, and other random output generators to generate the key. 
