\chapter{Common Scrambling Algorithm} \label{ch:CSA}
The CSA is currently the most commonly used encryption algorithm in DVB 
for encryption of video-streams. There are two versions of the DVB-CSA, 
CSA1 and CSA2, where the key-length is the only difference between them 
\citep[p. 23]{DVBScene:2013}. 

The CSA uses a combination of a block cipher, taking an input of a 
64-bit block, and a stream cipher. Both of the ciphers use the same 
key, so that the entire system uses the same key 
\citep[pp. 271--272]{WeiLi:2007}. This means that the complete 
algorithm would break if the key would be recovered. Using the same key 
does on the other hand allow us to easily change the key at regular 
intervals. 

CSA has been the official scrambling method for DVB since may 1994. CSA 
was to be easily implemented in hardware and hard to implement in 
software to make reverse-engineering of the algorithm difficult 
\citep{DVBScene:2013}.

%\section{History} \label{sec:History}
%CSA was largely kept secret until 2002, possibly since it was designed to be hard to reverse-engineered. The patent papers gave some hints of the layout, but important details like the layout of the S-boxes remained secret. Without the S-boxes, free implementations of the algorithm were out of question. Initially, CSA was to remain implemented in hardware only, but software implementations were found on the internet, which made it possible to analyze the entire solution.

%In 2002 FreeDec was released, implementing CSA in software. Though released as binary only, disassembly revealed the missing details and allowed reimplementation of the algorithm in higher-level programming languages. \Warning[source]{Is this relevant, and you need another source than Wikipedia}

%With CSA now publicly known in its entirety, cryptanalysts started looking for weaknesses.

\begin{figure}
  \begin{center}
    \includegraphics{blockcipher}
  \end{center}
  \caption{Number of bits in key used}
  \label{fig:blockcipher}
\end{figure}

\section{Why do we need a new standard?}
The DVB-CSA standard offers short-term protection (it assumes content 
is viewed in real time and not stored). Due to the development of how 
content is viewed during recent years, we now primarily need to be able 
to distribute content across homes. This means that the focus needs to 
be moved from securing delivery to securing content. \citep{Farncombe}

Another thing to bear in mind is the fact that more CPU-based units, 
such as smart-phones, tablets and computers are used to access contents 
now more than ever. In order to allow for descrambling on CPU-based 
units, a software-friendly scrambling algorithm might be needed.

\section{Layout of the CSA}
The CSA consists a block cipher and a stream cipher connected in 
sequence \citep[p. 271]{WeiLi:2007}. The block cipher reads 64-bit 
blocks of data, which is then run in Cipher Block Chaining-mode. The 
block cipher processes these blocks of data in 56 rounds. The output of 
this is sent to the stream cipher where additional encoding is 
performed. The first block of data sent from the block cipher to the 
stream cipher is used as an IV for the stream cipher, and is not 
encoded in this phase. \citep{DVBAnalysis:2006}

\section{Security}
One of the problems associated with CW distribution is the fact that CW 
sharing has become rather common \citep{Farncombe}. This is possibly 
due to the fact that the CW is sent in the clear between the smart card 
and the STB, meaning that a user might grab the clear CW during 
transmission and redistribute it over the internet. This has become a 
financial problem for content distributors, since people stop paying 
for the content which they are watching.

One way of dealing with CW sharing is to decode the encrypted CW on the 
CI system, and then encrypt it once again on the CI, before sending it 
to the STB. The latter key is setup between the CI system and the STB  
through a one time sychronization. This means that users are not able 
to grab the clear CW and redistribute it. \citep[pp. 12--13]{HIS:2011}

Another security issue that you need to think of when designing the 
hardware, to prevent content theft, is to make sure that no contacts 
are ever accessible from the top layer of the circuit board. This is 
due to the fact that people would be able to connect hardware to the 
board and download the material that way, if they were. 
\Warning[Source]{Except from Patrik Lantto}

We also need to be aware of people trying to break the algorithm 
through forced ways as well as CW sharing and hardware methods of 
stealing content.

%\subsection{Breaking the CSA}

There are a few standard ways to try, when you want to break a cipher. 
Those are the brute force-, known plaintext-, chosen plaintext- and 
birthday attacks \citep[pp. 31-34]{Schneier:2003}. You choose what 
method to use depending on what the cipher looks like. I will not 
discuss all of them, but I will talk about the most relevant ones for 
CSA here.

\subsection{Brute force}
The CSA uses a key consisting of 64-bits, which gives us 18.5 
Quintillion possible keys (Quintillion is $10^{18}$). But byte 3 and 7 
are often used as parity bytes in CA systems which leads to only 48 
bits being used in the key \citep{Breaking:2012}. This can be seen in 
figure \ref{fig:blockcipher}. 48 bits on other hand leads to $2^{48}$ 
combinations, which corresponds to 281 trillion possible keys (Trillion 
is $10^{12}$). Testing a million keys per second is about what is 
possible through on a modern x86 processor using software methods
\Warning[Todo]{How did I get this number?}, which means it would take 
roughly 3258 days to force brake the keys, which translates into 
roughly 8.8 years.

Moreover, systems need to change the key at least every 120 seconds 
\citep{Simpson:2009} and most systems issues new keys every 10-120 
second \citep{Wirt:2004}.

It is possible to use dedicated hardware and FPGA implementations to 
speed this up, using hardware accelerations and other methods. But even 
if we would be able to scan through 2.8 trillion keys per second, 
precisely allowing us to be certain to find the key in two minutes, we 
could just change the key more often. As such, the brute force method 
of obtaining the key is not a feasible option.

\subsection{Known plaintext attack}
The known plaintext attack is performed to find out the key, which can 
then be used to decrypt following ciphertexts. To be able to try this, 
a known plaintext-ciphertext pair is needed. You can try to find the keyif you have the both of them. This is done by identifying ciphertexts 
known to correspond to zero-filled plaintexts when breaking CSA 
\citep{Breaking:2012}. Then memories filled with precalculated keys are 
used to find which key the current pair corresponds to. This method is 
supposed to recover a key in roughly 7 seconds with a 97\% certainty 
according to \citet{Breaking:2012}.
