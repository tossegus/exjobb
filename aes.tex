\chapter{Advanced Encryption Standard}\label{ch:AES}
Both CSA3 and CISSA use the block cipher called AES, which will be 
explained in this chapter. The standard was determined by NIST in 
November 2001. AES is a symmetric block cipher which uses itself of
key lengths of either 128, 192 or 256 bits. It is based on an 
SP-network which is fast in both hardware and software. 

\section{Introduction}
The Rijndael cipher, which is used in AES, has key-sizes of at least 
128 bits. The block length is 128 bits. It uses 8 to 8 bit S-boxes 
and a encryption is made with a minimum of 10 rounds of repetition 
\citep[p. 79]{Stinson:2006}. A round can corresponds to one iteration 
of a certain part of the algorithm, and uses a subkey to the provided 
key. The keys used in the rounds are called round keys. It is a 
symmetric-key algorithm with a fixed block size of 128 bits, where the 
key-size can vary between 128, 192 or 256 bits. The number of cycles 
needed to convert the plaintext into ciphertext depends on the size of 
the key. The 128-bit key requires 10 cycles of repetitions (rounds). 
The 192-bit key requires 12 rounds and the 256-bit key requires 14 
rounds. \citep[p. 103]{Stinson:2006}

\section{Method}
The AES consists of a number of steps that are repeated for each block 
to be encoded. All of the steps are explained later in this chapter.
The steps to be performed are, according to \citet{Stinson:2006}:

\newpage
\emph{Set-up steps}
\begin{enumerate}
\item KeyExpansion - Produce round keys.
\item InitialRound - Combine each byte of the state with a byte of 
  round key.
\end{enumerate}
\emph{Steps performed during the rounds}
\begin{enumerate}
\item SubBytes - Each byte is \emph{substituted} using the Rijndael's 
  S-box.
\item ShiftRows - The rows of the state matrix are \emph{permutated}.
\item MixColums - The columns of the state matrix are multiplicated with a 
  matrix.
\item AddRoundKey - The state matrix is once again combined with 
  round-keys.
\end{enumerate}
In the final round everything except the MixColumns step are 
performed. Meaning that SubBytes, ShiftRows and AddRoundKey are 
performed.

The ciphertext is then defined as the state-matrix 
\citep[p. 103]{Stinson:2006}. As mentioned in section 
\ref{ch:ConfDiff} (\nameref{ch:ConfDiff}), both confusion and 
diffusion are nescessary to ensure a secure encryption. They can be 
seen in the SubBytes and ShiftRows steps above. These steps also 
performs whitening, which strengthens the cipher. Whitening is, as 
mentioned in \ref{ch:ConfDiff}, performed through an XOR between the 
roundkey and the data.

The KeyExpansion is explained in section \ref{sec:KeySch}.

Before anything can be done, the data needs to be put into a 
state-matrix, which can be seen in figure \ref{matrix2:state}.

\begin{figure}[h!]
  \begin{equation}
    \begin{bmatrix}
      a_{1, 1} & a_{1, 2} & a_{1, 3} & a_{1, 4} \\
      a_{2, 1} & a_{2, 2} & a_{2, 3} & a_{2, 4} \\
      a_{3, 1} & a_{3, 2} & a_{3, 3} & a_{3, 4} \\
      a_{4, 1} & a_{4, 2} & a_{4, 3} & a_{4, 4}
    \end{bmatrix}
  \end{equation}
  \caption{State-Matrix}
  \label{matrix2:state}
\end{figure}

\subsection{InitialRound}
This is an initial AddRoundKey which is explained in section 
\ref{sec:AddRoundKey}.

\subsection{SubBytes}
In the SubBytes step, each byte is sent to a Rijndael S-box (which is 
basically a lookup table, see figure \ref{matrix:rijSbox} in appendix 
\ref{app:matrix}) where they are substituted in a non-linear fashion. 
This gives us a substituted state matrix.

\subsection{ShiftRows}
The next step is called the ShiftRows step, which left-shifts the rows 
n-1 steps where n is the index of the row. This means that the first 
row is left as it is, the second row is shifted one step, the third 
row is shifted two steps, and the fourth row is shifted three steps.
The data is shifted cyclically, meaning that data which is shifted 
out of the left side of the state-matrix is shifted back in from the 
right side.

\subsection{MixColumns} \label{sec:MixColumns}
All of the multiplications performed in the MixColumns steps are 
take place in the Galois Field, which is why a lot of it might seem
illogical at first. 

In the MixColumns step, the four bytes of each row are combined 
through a matrix multiplication. The MixColumns function takes four 
bytes as input and multiplies them with a fixed matrix (figure 
\ref{matrix:rijMix} in appendix \ref{app:matrix}). While this might 
seem simple to do, it actually is not. The multiplication makes sure 
that each input byte affect all output bytes. \citep{Angelfire}

The matrix is multiplicated with the vector from the left, (4x4*4x1 = 
4x\hcancel{4*4x}1 = 4x1) where the vector is a column from the 
state-matrix. Multiplication with 1 means that the value is left 
untouched. Multiplication by 2 means left shift, then an XOR with 0x1B 
if the shifted value exceeds 0xFF. Multiplication with 3 is done in 
the same way as a multiplication with 2, except that the result after 
the shift and conditional XOR are then XOR:ed with the input value of 
the multiplication. All of the resulting values are then XOR:ed, 
leaving us with the result. All additions are replaced with XOR, since 
the calculations take place in the Galois Field (GF(\(2^8\))).

\subsection{AddRoundKey}\label{sec:AddRoundKey}
Each of the 16 bytes of the state are combined with a byte from the 
round key using a bitwise XOR. They are then combined to a state 
matrix (Figure \ref{matrix:state} in Appendix \ref{app:matrix}) 
containing 4x4 bytes.

\section{KeyExpansion}\label{sec:KeySch}
To generate round keys from the provided key, the Rijndael's key 
schedule is used. The schedule consists of a couple of loops and a key-
schedule core. The schedule core is the part that branches out if c 
modulo 16 is zero. The flowchart explaining the entire KeyExpansion 
can be viewed in Figure \ref{img:keysch} in Appendix \ref{app:fig}. 
To change the key schedule to fit a key size of 192 bits, you simply 
change the value that c is compared to in the first branch in the 
flowchart from 176 to 206.

This is done since AES requires a separate 128-bit (16-byte) 
round key for each round, plus one extra key for the initialization 
which means that the AES-128 requires 176 bytes, since AES-128 consists
of 10 rounds.

\subsection{Key-schedule core}\label{sec:kCore}
The key-schedule core takes an input of 4 bytes (32 bits) which it 
then rotates 1 byte (8 bits) to the left. Let us say that our key is 
\emph{AB CD EF 01}. This would give us the key \emph{CD EF 01 AB} 
after the rotation. This operation is also called the RotWord-
operation \citep[p. 107]{Stinson:2006}. The next step is to apply 
Rijndael's S-box to each of these bytes, giving us 4 new bytes. The 
bytes {AB CD EF 01} would give us {62 BD DF 7C}, when substituted 
according to the Rijndael S-box (Figure \ref{matrix:rijSbox} in 
Appendix \ref{app:matrix}).

The left-most byte is then XOR:ed with a value from the Rcon function 
depending on what round you are currently processing. You can read 
more about the Rcon function in section \ref{ch:Rcon}.

\subsection{Rijndael's S-Box}
Rijndael's S-box takes an input byte which it transforms according to 
a LUT (Figure \ref{matrix:rijSbox} in Appendix \ref{app:matrix}). 
Where the most significant nibble is located on the Y axis, and the 
least significant nibble is located on the X axis of the table. Given 
the input \emph{0x31}, the output \emph{0xC7} would be received from 
the Rijndael's S-box.

\subsection{Rcon} \label{ch:Rcon}
The value input into the Rcon function depends on what round you are 
currently at. Which means that you would choose Rcon(1) for the first 
round, Rcon(2) for the second round, and so on. The values in the Rcon 
array are calculated mathematically, but might as well be accessed 
from a vector, such as the one found in Figure \ref{matrix:rcon} in 
Appendix \ref{app:matrix}.

The steps to be performed in the Rcon function are illustrated in a
flowchard and can be viewed in figure \ref{img:rcon} in appendix 
\ref{app:fig}.

If the input value is 0, the output value is 0, otherwise the 
following steps are performed \citep{AES:2001} %\citep{RijndaelKeySchedule}. 
This can also be replaced by an S-box where you input your byte, and 
get another back, since the input byte is just used as a counter that 
decides how many times you perform steps \ref{item:step1} through 
\ref{item:step3} 

\begin{enumerate}
\item Set a variable c to 0x01.
\item If the input-value does not equal 1, set variable b to c 
  \& 0x80. Otherwise, go to \ref{item:exit_loop}. 
  \label{item:step1}
\item Left shift c one step.
\item If b is equal to 0x80 proceed to \ref{item:step2}, otherwise go to
  \ref{item:step3}.
\item Store the result of a bitwise XOR between c and 0x1B in c.
  \label{item:step2}
\item Decrease the input value by one, then go back to  
  \ref{item:step1}.
  \label{item:step3}
\item The output is set to c.
  \label{item:exit_loop}
\end{enumerate}
